%*******************************************************
% Abstract
%*******************************************************
%\renewcommand{\abstractname}{Abstract}
%\pdfbookmark[1]{Abstract}{Abstract}
%\begingroup
%\let\clearpage\relax
%\let\cleardoublepage\relax
%\let\cleardoublepage\relax
\chapter*{Abstract}
Embracing the digital information age, mankind is collecting an ever growing amount of structured, semi- and unstructured data. 
Hence, computationally answering complex natural language questions over multiple data sources is a challenging research task demanding a solution composed of natural language processing, information retrieval and knowledge extraction abilities. 

Building a novel question answering (QA) system over heterogeneous knowledge sources requires various building blocks.
For instance, named entity extraction approaches over unstructured data such as blogs, news or RSS feeds to capture the semantic meaning of a particular document.
Unfortunately, existing building blocks lack comparable evaluation settings, performance or quality. 

In this work, we identify three main challenges -- \emph{respectively research gaps} -- and present solutions for building basic components as well as whole systems for semantic question answering.
\begin{enumerate}
\item 
Over the last decades, several billion Web pages have been made available on the Web. 
The ongoing transition from the current \emph{Document Web} of unstructured data to the \emph{Data Web} yet requires scalable and accurate approaches for the extraction of structured data in RDF (Resource Description Framework) from these websites.
We address this key step for bridging the \emph{Semantic Gap}, i.e., extracting RDF from text,  with several approaches.
Our knowledge base-agnostic framework AGDISTIS can efficiently detect the correct URIs for a given set of named entities.
Furthermore, we present CETUS, an approach for recognizing entity types to populate RDF knowledge bases. 
\todo[inline]{@Axel: CDCR raus schmeißen?}
%Finally, we address the problem of assigning a single URI to named entities which stand for the same real-object across documents but are not yet available in the reference knowledge base.
\item 
This need to bridge between Document Web and the structured data on the Data Web has led to the development of a considerable number of annotation tools. However, these tools are currently still hard to compare since the published evaluation results are calculated on diverse datasets and evaluated based on different measures
The resulting \emph{Evaluation Gap} is tackled by GERBIL, an evaluation framework for semantic entity annotation. The rationale behind our framework is to provide developers, end users and researchers with easy-to-use interfaces that allow for the agile, fine-grained and uniform evaluation of annotation tools on multiple datasets.
\item 
Finally, the decentral architecture behind the Web has led to pieces of information being distributed across data sources with varying structure. 
To close the arising \emph{Information Gap}, we introduce HAWK, a novel entity search approach for Hybrid Question Answering based on combining structured and unstructured data sources.
Moreover, we summarize existing solutions for semantic QA systems and propose an innovative architecture for self-improving, -healing and -wiring complex QA systems.
\end{enumerate}
\todo[inline]{Numbers? http://stats.lod2.eu/ http://www.worldwidewebsize.com/}
%\newpage
%
%\pdfbookmark[1]{Zusammenfassung}{Zusammenfassung}
%\begingroup
%\let\clearpage\relax
%\let\cleardoublepage\relax
%\let\cleardoublepage\relax
%\chapter*{Zusammenfassung}
%\vfill

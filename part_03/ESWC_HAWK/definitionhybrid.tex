\section{Hybrid Search Definition}
\label{sec:hybridsearchdefinition}
Bhagdev et al.~\cite{Bhagdev:2008:HSE} define hybrid search as follows:

\begin{definition}
Hybrid Search (Bhagdev et al.):
\begin{itemize}
\item The application of semantic (metadata-based) search for the parts of the user queries 
where metadata is available; 
\item The application of keyword-based search for the parts not covered by metadata.
\end{itemize}
\end{definition}

However, the emergence of new database types, e.g. in geographical datasets, and novel query formats like natural language queries and gesture-based searches demand a more general redefinition of hybrid search.
\begin{definition}
Hybrid Search is search that
\item[i)] can handle any input format without restrictions and
\item[ii)] answers user information needs based on at least two knowledge bases with heterogeneous information representation schemes.
\end{definition}

Following this definition, HAWK is a hybrid search engine able to answer natural language entity queries based on RDF data and full-text indexes.

%\item merge results and present ranked list of information
%\item provenance information can help to rank documents
%\item differentiate between knowledge and document retrieval
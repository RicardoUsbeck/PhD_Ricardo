\begin{comment}
\subsection{Initialization of $L$ and $R$}
In most approaches, the matrices $L$ and $R$ are initialized by using random values.
While this approach still leads to converging results in practical applications, it is unsatisfactory as the results of the factorization may be different across different initializations.
We address the initialization problem by using a correlation-based approach.
The intuition behind our approach is that the latent features describe a space whose basic vectors stand for weakly correlated dimensions. 
We can thus initialize the matrices $L$ (and analogously the matrix $R$) by detecting the rows (resp. columns) that display the smallest correlation to other rows and using those as our initialization for $L$.
We go about implementing this intuition as follows: Let 
\begin{equation}
B = M \times M^\top \mbox{ and } A = \frac{B}{\max\limits_{i, j} b_{ij}}.
\end{equation}
The matrix $A$ encompasses how correlated the rows in $M$ are.
We assign each row $a_{i}$ a weight $w_i = \sum_{j=1}{n} a_{ij}$.
The weight $w_i$ tells us how correlated $a_i$ is to $a_1 \ldots a_n$.
We now sort the rows according to their weights in ascending order and select the r rows with the smallest weight.
These rows are then used to initialize $L$, ergo $l_1 = a_k$ with $\forall i \neq k, w_i \geq w_k$.  

The matrix $R$ can be initialized analogously by setting 
\begin{equation}
B = M^\top \times M \mbox{ and } A = \frac{B}{\max\limits_{i, j} b_{ij}}.
\end{equation}
\todo{Add example}

\begin{itemize}
\item use FOX to do NER over several documents \todo{No need for FOX in the experiments. Simply use the benchmarks directly.} 
\todo{Now it is implemented and getting back to the original work would be cumbersome.} 
\item describe an entity by its neighbors, i.e., window size and latent features
\item use cosine similarity as a baseline
\item cluster those entities across documents using Axels framework
\item named entities are the same resources if they end up in the same cluster...\todo[inline]{RU: I fear the noise hear when number of clusters is too low}
\item formulate good URIs for them, see http://www.w3.org/TR/cooluris/

\end{itemize}

LR^T = 

  5,103 1,896 -0,712 1,555
  3,412 1,274 -0,450 1,085
  1,540 1,040  1,780 3,949
  1,169 0,799  1,394 3,073
 -0,438 0,543  3,076 5,125
\end{comment}
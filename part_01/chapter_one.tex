\todo[inline]{http://kidehen.blogspot.de/2015/09/what-happened-to-semantic-web.html Are we there yet?}
\chapter{Introduction}
This chapter is partially based on the authors PhD symposium paper~\cite{combiningLDandIR} and presents the overall motivation, structure and scientific contribution of this thesis.
\section{Motivation}
Since the proposal of hypertext by Tim-Berners Lee to his employer CERN at March 12, 1989\footnote{\url{http://www.w3.org/People/Berners-Lee/Longer.html}} the World Wide Web gathered more than one billion web pages and still grows\footnote{\url{http://www.internetlivestats.com/total-number-of-websites/}}.
With the proposed Semantic Web vision~\cite{bernerslee2001semantic}, Lee et al. extended the existing Web to allow for better reuse, sharing and understanding of data by introducing novel W3C\footnote{World Wide Web consortium \url{http://www.w3.org/}} standards. 
%taken from HAWK

With the growing amount of data and the ongoing decentralisation of knowledge, intuitive ways for non-experts to access this data are required. 
To access this data, search engines trained humans to find information by typing keywords and to expect several web documents in return~\cite{ilprints361}.
Yet, humans think and communicate their information needs commonly in their natural language rather than using incoherent keyword phrases~\cite{woods1973progress}. 
Hence, answering complex questions often required combining information from various, differently structured data sources.
Therefore, \emph{question answering (QA) systems} provide an easy and efficient way to query data via a known and complex input language, reducing a possible loss of precision and time while reformulating the search intention to transform it into a machine-readable way.
Furthermore, QA systems enable answering natural language queries with concise results instead of verbose links to web documents. 

\todo[inline]{introduce the need for KE and benchmarking it}

%Already in 1961, the Baseball system~\cite{green1961baseball} by Green et al. identified natural language as the most effective and convenient way for men to communicate with the growing amount of computer-centered systems. 

Building a question answering (QA) system over heterogeneous knowledge sources requires various building blocks.
In this work, we identify three main challenges -- \emph{respectively research gaps} -- and present solutions for building basic components as well as a complete system for hybrid semantic question answering.
\begin{enumerate}
\item 
Over the last decades, several billion Web pages have been made available on the Web. 
The ongoing transition from the current \emph{Document Web} of unstructured data to the \emph{Data Web}, a first step towards realizing the Semantic Web vision, yet requires scalable and accurate approaches for the extraction of structured data in RDF (Resource Description Framework)\footnote{\url{http://www.w3.org/standards/techs/rdf\#w3c_all}} from these websites.
We address this key step for bridging the \emph{Semantic Gap}, i.e., extracting RDF from text,  with several approaches.
Our knowledge base-agnostic framework AGDISTIS~\cite{agdistis_iswc} can efficiently detect the correct URIs for a given set of named entities.
Furthermore, we present CETUS~\cite{CETUS_2015}, an approach for recognizing entity types to populate RDF knowledge bases. 
\todo[inline]{@Axel: CDCR raus schmeißen?}
%Finally, we address the problem of assigning a single URI to named entities which stand for the same real-object across documents but are not yet available in the reference knowledge base.
\item 
The ongoing research on closing the Semantic Gap yield a large number of annotation tools.
However, these tools are currently still hard to compare since the published evaluation results are calculated on diverse datasets and evaluated based on different measures.
The resulting \emph{Evaluation Gap} is tackled by GERBIL~\cite{gerbil}, an evaluation framework for semantic entity annotation. The rationale behind our framework is to provide developers, end users and researchers with easy-to-use interfaces that allow for the agile, fine-grained and uniform evaluation of annotation tools on multiple datasets.
\item 
Finally, the decentral architecture behind the Web has led to pieces of information being distributed across data sources with varying structure. 
The search functionality to be developed in this thesis is going to be \emph{hybrid}, i.e., simultaneously performing a search on full-texts and semantic knowledge bases.
Different entity search algorithms need to be developed based on the significantly different data structures and problems arising from them. 
To close the arising \emph{Information Gap}, we introduce HAWK~\cite{hawk_2015}, a novel entity search approach for Hybrid Question Answering based on combining structured and unstructured data sources.
Moreover, we summarize existing solutions for semantic QA systems and propose an innovative architecture for self-improving, -healing and -wiring complex QA systems.
\end{enumerate}

\section{Thesis Structure}

This thesis is structured in four parts which reflect the efforts and challenges of the last three years of PhD work. 
Every part is based on several peer-reviewed publications by the author of this work. 
Each section will indicate the related publications as well as the contribution of the thesis author to these publications. 
The author works and marks as thoroughly as possible all parts from other works. 
However, if there should be any mistakes or left-out citations the author will immediately after discovery publish an on-line erratum clarifying the mistake. 

Due to the wide research areas covered in but required for this thesis, the author decided to partition this work into four parts, namely the introduction, knowledge extraction, hybrid question answering and the appendix. 

\subsection{Part 1: Introduction}
In the first part, the reader will become familiar with the problem and the research challenges occurring while developing a novel hybrid question answering system as well as tools for extracting semantic knowledge from non-structured data and benchmarking existing semantic annotation approaches.
Furthermore, we introduce the basic standards, technologies and formalities to understand the research problems and solutions as well as the notation used throughout this work.

\subsection{Part 2: Knowledge Extraction}



\subsection{Part 3: Hybrid Question Answering}

\subsection{Part 4: Appendix}

From now on, the author will use the we-form as narrative.

\section{Scientific Contributions}
This PhD work is dimensioned for three years. 
After intense literature reviews in the beginning of the first year the need for annotated Web data has been identified.
As a logical consequence, the development of AGDISTIS and REX had been finished by the end of the first year. 
Alongside, a gold standard ($N^3$) has been created to be able to evaluate the approaches mentioned above.

The second year will be used for developing and assessing the corresponding search and ranking procedures. 
To measure the quality of the \emph{auto-completion} technology, we assess different real-world query logs from our industry partner.
Thereby, we analyze how much characters are need to understand the query correct.
Additionally, we focus on the efficiency of the system in terms of milliseconds to react on a pressed key.

Considering the ranking evaluation, we will use standard precision, recall and f-measures as well as rank comparision measures, e.g., mean reciprocal rank. 
The underlying data is provided by the industry partner through human rater assessments and several comparisons to real-life search engines, e.g., Google or Wolfram Alpha.

Afterwards, the combined pipeline itself will be evaluated in a qualitative study using professionals and end users.
Therefore, empirical methods like Likert-scale questionnaires and direct relevance feedback will be used.


Next to refining already submitted work and optimizing the source code to meet industrial production standards, the developed approaches and algorithms will be refined in a spiral way if unpredictable results occur.
Thereby, upcoming ideas will be interweaved with the presented schedule creating a closed loop consisting of research question, development, evaluation and new research questions.
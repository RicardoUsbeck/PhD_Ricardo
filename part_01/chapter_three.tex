
\section{Related Work}
\label{sec:relatedwork}
AGDISTIS is related to the research area of Information Extraction~\cite{nad:sek} in general and to \ac{NED} in particular.
Several approaches have been developed to tackle \ac{NED}. 
Cucerzan presents an approach based on extracted Wikipedia data towards disambiguation of named entities~\cite{Cucerzan07}.
The author aims to maximize the agreement between contextual information of Wikipedia pages and the input text by using a local approach.
%\todo{What's a local approach?}
\emph{Epiphany}~\cite{epiphany} identifies, disambiguates and annotates entities in a given HTML page with RDFa. 
Ratinov et al.~\cite{rat:rot} described an approach for disambiguating entities from Wikipedia. 
The authors argue that using Wikipedia or other ontologies can lead to better global approaches than using traditional local algorithms which disambiguate each mention separately using,\,e.g., text similarity. %for word sense disambiguation.
Kleb et al.~\cite{Kleb11WIMS,KlebESWC10} developed and improved an approach using ontologies to mainly identify geographical entities but also people and organizations in an extended version. 
These approaches use Wikipedia and other Linked Data \ac{KB}s.
LINDEN~\cite{LINDEN} is an entity linking framework that aims at linking identified named entities to a knowledge base.
To achieve this goal, LINDEN collects a dictionary of the surface forms of entities from different Wikipedia sources, storing their count information.

Wikipedia Miner~\cite{milne2008learning} is the oldest approach in the field of \emph{wikification}.
Based on different machine learning algorithms, the systems disambiguates w.r.t. prior probabilities, relatedness of concepts in a certain window and context quality. 
The authors evaluated their approach based on a Wikipedia as well as an AQUAINT subset. 
Unfortunately, the authors do not use the opportunities provided by Linked Data like DBpedia.

Using this data the approach constructs candidate lists and assigns link probabilities and global coherence for each resource candidate.
The AIDA approach~\cite{AIDA} for \ac{NED} tasks is based on the YAGO2 \ac{KB} and relies on sophisticated graph algorithms. 
Specifically, this approach uses dense sub-graphs to identify coherent mentions using a greedy algorithm enabling Web scalability. 
Additionally, AIDA disambiguates w.r.t.~similarity of contexts, prominence of entities and context windows.

Another approach is DBpedia Spotlight~\cite{spotlight}, a framework for annotating and disambiguating Linked Data Resources in arbitrary texts.
In contrast to other tools, Spotlight is able to disambiguate against all classes of the DBpedia ontology.
Furthermore, it is well-known in the Linked Data community and used in various projects showing its wide-spread adoption.\footnote{\url{https://github.com/dbpedia-spotlight/dbpedia-spotlight/wiki/Known-uses}}
Based on a vector-space model and cosine similarity DBpedia Spotlight is publicly available via a web service\footnote{\url{https://github.com/dbpedia-spotlight/dbpedia-spotlight/wiki/Web-service}}.

In 2012, Ferragina et al. published a revised version of their disambiguation system called TagMe 2.
The authors claim that it is tuned towards smaller texts,\,i.e., comprising around 30 terms.
TagMe 2 is based on an anchor catolog (\texttt{<a>} tags on Wikipedia pages with a certain frequency), a page catalogue (comprising all original Wikipedia pages,\,i.e., no disambiguations, lists or redirects) and an in-link graph (all links to a certain page within Wikipedia).
First, TagMe 2 identifies named entities by matching terms with the anchor catalog and second disambiguates the match using the in-link graph and the page catalog via a collective agreement of identified anchors. 
Last, the approach discards identified named entities considered as non-coherent to the rest of the named entities in the input text.  

In 2014, Babelfy~\cite{babelfy} has been presented to the community.
Based on random walks and densest subgraph algorithms Babelfy tackles \ac{NED} and is evaluated with six datasets, one of them the later here used AIDA dataset. 
In constrast to AGDISTIS, Babelfy differentiates between word sense disambiguation, i.e., resolution of polysemous lexicographic entities like \emph{play}, and entity linking, i.e., matching strings or substrings to knowledge base resources.
Due to its recent publication Babelfy is not evaluated in this paper.

Recently, Cornolti et al.~\cite{cornolti} presented a benchmark for \ac{NED} approaches.
The authors compared six existing approaches, also using DBpedia Spotlight, AIDA and TagMe 2, against five well-known datasets. % on different tasks and with different measures.
Furthermore, the authors defined different classes of named entity annotation task, e.g. \emph{`D2W'}, that is the disambiguation to Wikipedia task which is the formal task AGDISITS tries to solve.
We consider TagMe 2 as state of the art w.r.t. this benchmark although only one dataset has been considered for this specific task.
%We analyze the performance of DBpedia Spotlight, AIDA, TagMe 2 and our approach AGDISTIS on four of the corpora from this benchmark in Section~\ref{sec:eval}.
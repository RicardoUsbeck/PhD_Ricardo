\chapter{Related Work}

\graffito{This chapter introduces the state of the art in the research areas related to this thesis. This overview is based on the publications	\cite{rex,qasurvey, pool-foundations-lod,n3, gerbil_dev_2015,CETUS_2015, combiningLDandIR, HAWK_CLEF_2015, agdistis_ecai,agdistis_iswc, agdistisdemo,HAWK_NLIWOD_2015, hawk_2015, gerbil_demo_2015, GERBIL}}


In this chapter, we give an overview of related research required to bridge the Semantic Gap, i.e., the areas of \ac{RDF} type annotation, \ac{NED} and extraction of semantic data from templated websites, as well as the Evaluation Gap and the Information Gap.

%%%%%%%%%%%CETUS
\section{Semantic Gap - RDF Extraction}
\subsection{RDF Type Annotation}

In 2015, the {Open Knowledge Extraction Challenge} had its first installation at the Extended Semantic Web Conference~\cite{okechallenge}.
Within this challenge, the second task was to identify entity types (\texttt{rdf:type} information) for a given entity inside a given text and linking this type to a \ac{KB}, i.e., to the subset of the DOLCE+DnS Ultra Lite ontology classes\footnote{\url{http://www.ontologydesignpatterns.org/ont/dul/DUL.owl}}.

The winner of the challenge was CETUS~\cite{CETUS_2015}, a pattern-based approach inspired by  Hearst Patterns~\cite{Hearst1992} and presented in Chapter~\ref{cha:cetus}.
Those Hearst Patterns match text parts describing hyponym relations between two nouns.
In difference to CETUS' patterns, the Hearst Patterns have been extracted from a large corpus using a bootstrapping approach.
As described in Section~\ref{sec:patternExt}, our patterns are defined for matching text parts describing the type relation of a given entity and have been created manually during an iterative, incremental process.

Another approach is FRED~\cite{fred_typing}, which is a machine reader, i.e., a holistic approach towards representing a whole text as \ac{RDF}. 
Here, FRED reuses various tools like Boxer~\citep{Bos:2008:WSA:1626481.1626503} and TagMe2~\cite{TagMe2} to be able to spot not only the common types like person, organizations and places but also events.

The third challenge participant was implemented by Gao et al.~\cite{oak_sheffield}. 
Their unsupervised system for entity typing is based on a rich feature set, e.g., words, named entities, gazetteer tokens and semantic distance, to identify the type evidence string. 
Next, they match and align the string to DBpedia before transferring this information to the DOLCE+DnS Ultra Lite ontology.
For non-matchable entities, the authors compute a semantic distance and match the closest possible type class.  


%Next to the above mentioned challenges about entity linking, several tools have been introduced with the ability to type entities, e.g., FOX~\cite{FOX}.


%%%%%%%AGDISTIS
\subsection{Named Entity Disambiguation}

The research area of Information Extraction~\cite{nad:sek} in general and to \ac{NED} in particular.
Several approaches have been developed to tackle \ac{NED}.
\bigskip


Cucerzan presents an approach based on extracted Wikipedia data towards disambiguation of named entities~\cite{Cucerzan07}.
The author aims to maximize the agreement between contextual information of Wikipedia pages and the input text by using a local approach.
%\todo{What's a local approach?}

%{Epiphany}~\cite{epiphany} identifies, disambiguates and annotates entities in a given HTML page with RDFa. 

Ratinov et al.~\cite{rat:rot} described an approach for disambiguating entities from Wikipedia. 
The authors argue that using Wikipedia or other ontologies can lead to better global approaches than using traditional local algorithms which disambiguate each mention separately using,\,e.g., text similarity. %for word sense disambiguation.

Kleb et al.~\cite{Kleb11WIMS,KlebESWC10} developed and improved an approach using ontologies to mainly identify geographical entities but also people and organizations in an extended version. 
These approaches use Wikipedia and other \ac{LD} \ac{KB}s.

LINDEN~\cite{LINDEN} is an \ac{EL} framework that aims at linking identified named entities to a \ac{KB}.
To achieve this goal, LINDEN collects a dictionary of the surface forms of entities from different Wikipedia sources, storing their count information.

Wikipedia Miner~\cite{milne2008learning} is the oldest approach in the field of {wikification}.
Based on different machine learning algorithms, the systems disambiguates w.r.t. prior probabilities, relatedness of concepts in a certain window and context quality. 
The authors evaluated their approach based on a Wikipedia as well as an AQUAINT subset. 
Unfortunately, the authors do not use the opportunities provided by \ac{LD} like DBpedia.

Using this data the approach constructs candidate lists and assigns link probabilities and global coherence for each resource candidate.
The AIDA approach~\cite{AIDA} for \ac{NED} tasks is based on the YAGO2 \ac{KB} and relies on sophisticated graph algorithms. 
Specifically, this approach uses dense sub-graphs to identify coherent mentions using a greedy algorithm enabling Web scalability. 
Additionally, AIDA disambiguates w.r.t.~similarity of contexts, prominence of entities and context windows.

Another approach is DBpedia Spotlight~\cite{spotlight}, a framework for annotating and disambiguating \ac{LD} resources in arbitrary texts.
In contrast to other tools, Spotlight is able to disambiguate against all classes of the DBpedia ontology.
Furthermore, it is well-known in the \ac{LD} community and used in various projects showing its wide-spread adoption.\footnote{\url{https://github.com/dbpedia-spotlight/dbpedia-spotlight/wiki/Known-uses}}
Based on a vector-space model and cosine similarity DBpedia Spotlight is publicly available via a web service\footnote{\url{https://github.com/dbpedia-spotlight/dbpedia-spotlight/wiki/Web-service}}.

In 2012, Ferragina et al. published a revised version of their disambiguation system called TagMe 2.
The authors claim that it is tuned towards smaller texts,\,i.e., comprising around 30 terms.
TagMe 2 is based on an anchor catolog (\texttt{<a>} tags on Wikipedia pages with a certain frequency), a page catalogue (comprising all original Wikipedia pages,\,i.e., no disambiguations, lists or redirects) and an in-link graph (all links to a certain page within Wikipedia).
First, TagMe 2 identifies named entities by matching terms with the anchor catalog and second disambiguates the match using the in-link graph and the page catalog via a collective agreement of identified anchors. 
Last, the approach discards identified named entities considered as non-coherent to the rest of the named entities in the input text.  

In 2014, Babelfy~\cite{babelfy} has been presented to the community.
Based on random walks and densest subgraph algorithms Babelfy tackles \ac{NED} and is evaluated with six datasets, one of them the later here used AIDA dataset. 

In constrast to AGDISTIS, Babelfy differentiates between word sense disambiguation, i.e., resolution of polysemous lexicographic entities like \texttt{play}, and \ac{EL}, i.e., matching strings or substrings to \ac{KB} resources.
Due to its recent publication Babelfy is not evaluated in this paper.
We present our approach, dubbed AGDISTIS~\cite{agdistis_iswc}, in Chapter~\ref{cha:agdistis}. 
%Recently, Cornolti et al.~\cite{cornolti} presented a benchmark for \ac{NED} approaches.
%The authors compared six existing approaches, also using DBpedia Spotlight, AIDA and TagMe 2, against five well-known datasets. % on different tasks and with different measures.
%Furthermore, the authors defined different classes of named entity annotation task, e.g. \emph{`D2W'}, that is the disambiguation to Wikipedia task which is the formal task AGDISITS tries to solve.
%We consider TagMe 2 as state of the art w.r.t. this benchmark although only one dataset has been considered for this specific task.
%%We analyze the performance of DBpedia Spotlight, AIDA, TagMe 2 and our approach AGDISTIS on four of the corpora from this benchmark in Section~\ref{sec:eval}.


%%%REX

\subsection{RDF Extraction from Semi-Structured Websites and Wrapper Induction}

In Chapter~\ref{cha:rex}, we present our approach, dubbed REX, towards \ac{RDF} extraction from templated websites which relies on \ac{NED}, see above, and data validation based on automatically extracted axioms~\cite{Buehmann2012}. 
\bigskip

The extraction of axioms from \ac{KB}s using statistical information~\cite{Buehmann2012,pattern_enrichment} as also flourished over the last years. 
The main idea underlying these approaches is to use instance knowledge from  \ac{KB}s without expressive schemas to compute the axioms which underlie the said  \ac{KB}s. 
We refer the reader to the publications above for an overview of these two research areas.
\bigskip

Furthermore, the main related research area is wrapper induction. 
Early approaches to learning web wrappers were mostly supervised~\cite{flesca2004web,Hogue:2005:TAU:1060745.1060762}. 
These systems were provided with annotated pages out of which they infer extraction rules that allow extracting data from other unlabeled pages with the same structure as the annotated pages). 

For example, Hogue et al.~\cite{Hogue:2005:TAU:1060745.1060762} presents {Tresher}, a supervised system that allows non-technical end-users to teach their browser how to extract data from the Web. 
%The approach relies on labeled web pages and a distance function between DOM trees to generate extraction templates based on user annotations.
%While remaining able to generate high-precision wrappers thanks to the human supervision, 
Supervised approaches were yet deemed costly due to the human labor necessary to annotate the input web pages. 

Unsupervised wrapper induction methods have thus been explored~\cite{exalg,DBLP:journals/aai/CrescenziM08} to reduce the annotation costs. 
However, the absence of a supervision often lead these systems to produce wrappers of accuracy not suitable for production level usage.

Novel approaches thus aim to minimize the annotation costs while keeping a high precision.
For example, the approach presented by Davi et al.~\cite{Dalvi:2011:AWL:1938545.1938547} relies on the availability of a  \ac{KB} in the form of dictionaries and regular expressions to automatically obtain training data. 
%They present an inference algorithm to learn wrappers in presence of noise in the input data but neglect any issue that could arise from the presence of biased samples.

Recently, Crescenzi et al.~\cite{Crescenzi2013}describes a supervised framework that is able to profit from crowd-provided training data. 
The learning algorithm controls the cost of the crowd sourcing campaign w.r.t. quality of the output wrapper.
However, these novel approaches do not target the generated of \ac{RDF} data.

\ac{LD} has been used to learn wrappers to extract \ac{RDF} from the Web in recent years. 
For example, Gentile et al.~\cite{Gentile2013} exploits \ac{LD} as a training data to find instances of given classes such as universities and extract the attributes of these instances while relying on the supervised wrapper induction approach presented by Hao et al.~\cite{Hao2011}. However, they require a manual exploration of the \ac{LD} sources to generate their training data, which leads to a considerable amount of manual effort.

The {\sc Deimos} project~\cite{conf/aaaiss/ParundekarKA10} is similar to REX, as it aims at bringing to the Semantic Web the data that are published through the rest of the Web. 
However, it focuses on the pages behind web forms.


OntoSyphon~\cite{DBLP:journals/ws/McDowellC08} operates in an ontology-driven manner: taking any ontology as input, OntoSyphon uses the ontology to specify web searches that identify possible semantic instances, relations, and taxonomic information, in an unsupervised manner. However, the approach makes use of extraction patterns that work for textual documents rather than structured web pages.

To the best of our knowledge, none of the existing approaches covers all steps that are required to extract consistent \ac{RDF} from the Web. 
Especially, only Parundekar et al.'s work~\cite{conf/aaaiss/ParundekarKA10} is able to generate \ac{RDF} but does not check it for consistency.
In contrast, REX~\cite{rex} is the first approach that is scalable, low-cost, accurate and can generate consistent \ac{RDF}. 


%%%%%%%N3


\section{Evaluation Gap - Benchmarking Semantic Annotation Systems}

\ac{NER} and \ac{EL} have gained significant momentum with the growth of \ac{LD} and structured \ac{KB}s. 
Over the last few years, the problem of result comparability has thus led to the development of more than a dozen datasets and a hand full of frameworks.
\bigskip

However, most of the datasets are not freely available, e.g., the full CoNLL 2003 shared task~\cite{conll2003} used in~\cite{AIDA}.
Others are not yet annotated with \ac{LD} from DBpedia like the WePS (Web people search) evaluation dataset~\cite{WEPS}.

In 2013, Steinmetz et al.~\cite{NEDstatBench} published a statistical benchmark evaluation.
Three datasets are described there of which two are freely available\footnote{\url{http://www.yovisto.com/labs/ner-benchmarks/}}.
The authors also analyze the aim of each dataset according to four baseline algorithms and respective underlying dictionaries for NED.
Furthermore, the two available datasets, \texttt{KORE50} and \texttt{DBpedia Spotlight}, are published using NLP interchange format (NIF)~\cite{ISWC2013NIF} and have been part of the original benchmarks of~\cite{AIDA,spotlight}. 
Both datasets do not inherit a clear license nor do they contain a large number of documents typically needed for optimization problems.
\texttt{KORE50} comprises 50 sentences overall whereas \texttt{DBpedia Spotlight} consists of 58 sentences only. 

In contrast,  $\mbox{N}^3$ provides larger and more insightful datasets, published\footnote{\url{http://aksw.org/Projects/N3nernednif}} in NIF, to leverage the possibility of optimizing NER and NED algorithms via \ac{LD} and to ensure a maximal interoperability to overcome the need for corpus-specific parsers. 
All three datasets contain high-quality, manual annotations of named entities which are linked to the DBpedia~\cite{dbpedia-swj} \ac{KB}.
These datasets have already been used to evaluate~\cite{AIDA,spotlight} as well as ~\cite{GER+13,agdistis_iswc,GERBIL}.
We present $\mbox{N}^3$ in more detail in Chapter~\ref{cha:N3}.


%%%%%%%GERBIL
\bigskip

Over the course of the last 25 years several challenges, workshops and conference dedicated themselves to the comparable evaluation of information extraction (IE) systems. 
Starting in 1993, the Message Understanding Conference (MUC) introduced a first systematic comparison of information extraction approaches~\cite{Sundheim:1993:TIE:1072017.1072023}.
Ten years later, the Conference on Computational Natural Language Learning (CoNLL) started to offer a shared task on \ac{NER} and published the CoNLL corpus~\cite{conll2003}.
In addition, the Automatic Content Extraction (ACE) challenge~\cite{doddington2004automatic}, organized by NIST, evaluated several approaches but was discontinued in 2008. 
Since 2009, the text analytics conference hosts the workshop on \ac{KB} population (TAC-KBP)~\cite{mcnamee2009overview} where mainly linguistic-based approaches are published.
The Senseval challenge, originally concerned with classical NLP disciplines, has wided it focus in 2007 and changed its name to SemEval to account for the recently recognized impact of semantic technologies~\cite{kilgarri1998senseval}.
The Making Sense of Microposts workshop series (MSM) established in 2013 an entity recognition and in 2014 an \ac{EL} challenge thereby focusing on tweets and microposts~\cite{MSM2014}.
However, nearly each event chose a different set of benchmark datasets and settings. 
Lately, several benchmark frameworks were introduced.

In 2014, Carmel et al.~\cite{ERD2014} introduced one of the first Web-based evaluation systems for NER and NED and the centerpiece of the entity recognition and disambiguation (ERD) challenge. Here, all frameworks are evaluated against the same unseen dataset and provided with corresponding results. 

The BAT-framework~\cite{cornolti} is designed to facilitate the benchmarking of \ac{NER}, \ac{NED} -- also known as \ac{EL} -- and concept tagging approaches.
BAT compares seven existing entity annotation approaches using Wikipedia as reference.
Moreover, it defines six different task types, five different matchings and six evaluation measures providing five datasets.

Rizzo et al.~\cite{rizzo2014}  present a state-of-the-art study of NER and NEL systems for annotating newswire and micropost documents using well-known benchmark datasets, namely CoNLL2003 and Microposts 2013 for NER as well as AIDA/CoNLL and Microposts2014~\cite{Cano2014} for NED. 
The authors propose a common schema, named the NERD ontology\footnote{\url{http://nerd.eurecom.fr/ontology}}, to align the different taxonomies used by various extractors. To tackle the disambiguation ambiguity, they propose a method to identify the closest DBpedia resource by (exact-)matching the entity mention.

GERBIL~\cite{GERBIL} goes beyond the state of the art by extending the BAT-framework as well as~\cite{rizzo2014} in several dimensions to enhance reproducibility, diagnostics and publishability of entity annotation systems. 
In particular, GERBIL provides \numberOfadditionalDatasets additional datasets and \numberOfadditionalAnnotators additional annotators compared to the BAT-Framework. 
Moreover, the framework addresses the lack of treatment of NIL values within the BAT-framework and provides more wrapping approaches for annotators and datasets. 
Additionally, GERBIL provides persistent URLs for experiment results, unique URIs for frameworks and datasets, a machine-readable output and automatic dataset updates from data portals. 
Thus, it allows for a holistic comparison of existing annotators while simplifying the archiving of experimental results. 
GERBIL offers opportunities for the fast and simple evaluation of entity annotation system prototypes via novel NIF-based~\cite{NIF} interfaces, which are designed to simplify the exchange of data and binding of services.
In Chapter~\ref{cha:gerbil}, this thesis presents GERBIL in more detail.

%%%%%%%QA
\section{Information Gap - Hybrid Question Answering}

Hybrid \ac{QA} is related to the fields of hybrid search and \ac{QA} over structured data. 
In the following, we thus give a brief overview of the state of the art in these two areas of research.

First, we present hybrid search approaches which use a combination of structured as well as unstructured data to satisfy an user's information need. 
Semplore~\cite{Zhang:2007a} is the first known hybrid search engine by IBM.
It combines existing information retrieval index structures and functions to index \ac{RDF} data as well as textual data. 
Semplore focuses on scalable algorithms and is evaluated on an early \ac{QALD} dataset.

Bhagdev et al.~\cite{Bhagdev:2008:HSE} describe an approach to hybrid search combining keyword searches, Semantic Web inferencing and querying. 
The proposed K-Search outperforms both keyword search and pure semantic search strategies.
Additionally, an user study reveals the acceptance of the Hybrid Search paradigm by end users.
K-Search retrieves only documents where a full-text match and an ontology match via SPARQL is available, loosing possible matching documents.

A personalized hybrid search implementing a  hotel search service as use case is presented in~\cite{DBLP:journals/kbs/Yoo12}. 
By combining rule-based personal knowledge inference over subjective data, such as expensive locations, and reasoning, the personalized hybrid search has been proven to return a smaller amount of data thus resulting in more precise answers. 
Additionally, Yoo presents an architectures for hybrid search and a novel hotel ontology derived from crowd data. 
Unfortunately, the paper does not present any qualitative evaluation and it lacks source code and test data for reproducibility. 

All presented approaches fail to answer natural-language questions.
Besides keyword-based search queries, some search engines already understand natural language questions. \ac{QA} is more difficult than keyword-based searches since retrieval algorithms need to understand complex grammatical constructs.


Second, we explain several \ac{QA} approaches in the following.

Already in 1961, the Baseball system~\cite{green1961baseball} by Green et al. identified natural language as the most effective and convenient way for men to communicate with the growing amount of computer-centered systems. 

{Schlaefer et al.~\cite{ephyra2007}} describe {Ephyra}, an open-source \ac{QA} system and its extension with factoid and list questions via semantic technologies.
Using Wordnet as well as an answer type classifier to combine statistical, fuzzy models and previously developed, manually refined rules. The disadvantage of this system lies in the hand-coded answer type hierarchy. % which prohibits its multi-lingual use.

Cimiano et al.~\cite{orakel} developed {ORAKEL} to work on structured \ac{KB}s.
The system is capable of adjusting its natural language interface using a refinement process on unanswered questions. 
Using F-logic and SPARQL as transformation objects for natural language user queries it fails to make use of Semantic Web technologies such entity disambiguation.

{Lopez et al.~\cite{poweraqua}} introduce {PowerAqua}, another open source system, which is agnostic of the underlying yet heterogeneous sets of \ac{KB}s. 
It detects on-the-fly the needed ontologies to answer a certain question, maps the users query to Semantic Web vocabulary and composes the retrieved (fragment-)information to an answer. 

{Damljanovic et al.~\cite{freya}} present {FREyA} to tackle ambiguity problems when using natural language interfaces. 
Many ontologies in the Semantic Web contain hard to map relations, e.g., questions starting with 'How long$\ldots$' can be disambiguated to a time or a distance. 
By incorporating user feedback and syntactic analysis FREyA is able to learn the users query formulation preferences increasing the systems \ac{QA} precision. 

{Cabrio et al.~\cite{qakis}} present a demo of {QAKiS}, an agnostic \ac{QA} system grounded in ontology-relation matches. 
The relation matches are based on surface forms extracted from Wikipedia to enforce a wide variety of context matches, e.g., a relation birthplace(person, place) can be explicated by X was born in Y or Y is the birthplace of X. 
Unfortunately, QAKiS matches only one relation per query and moreover relies on basic heuristics which do not account for the variety of natural language in general.

{Unger et al.~\cite{pythia}} describe {Pythia}, a \ac{QA} system based on two steps.
First, it uses a domain-independent representation of a query such as verbs, determiners and wh-words.
Second, Pythia is based on a domain-dependent, ontology-based interface to transform queries into F-logic.


Moreover, Unger et al.~\cite{template} present a manually curated, template-based approach, dubbed {TBSL}, to match a question against a specific SPARQL query. 
Combining \ac{NLP} capabilities with \ac{LD} leads to good benchmark results on the \ac{QALD}-3 benchmark.
This approach cannot be used to a wider variety of natural language questions due to its limited repertoire of 22 templates.

{Shekarpour et al.~\cite{SINA_WebSemantic}} develop {SINA} a keyword and natural language query search engine which is aware of the underlying semantics of a keyword query. 
The system is based on Hidden Markov Models for choosing the correct dataset to query.
Underlying is a SPARQL generation process which means SINA is only capable of dealing with \ac{LD} and cannot benefit from the wealth of unstructured information in the current Web.


{Treo}~\cite{treo} emphasis the connection between the semantic matching of input queries and the semantic distributions underlying \ac{KB}s.
The tool provides an entity search, a semantic relatedness measure, and a search based on spreading activation.

Recently, Peng et al.~\cite{DBLP:journals/corr/PengZZ14} describe an approach for hybrid \ac{QA} mapping keywords as well as resource candidates to modified SPARQL queries. 

The winner of the recent \ac{QALD}-5 challenge is Xser~\cite{xu2014xser}.
First, this approach assigns semantic labels, i.e., variables, entities, relations and categories, to phrases by casting them to a sequence labeling pattern recognition problem which is then solved by a structured perceptron.
Second, it uses the target knowledge base to instantiate the generated template.
For instance, moving to another domain based on a different knowledge base thus only affects parts of the approach so that the conversion effort is lessened.


Several industry-driven \ac{QA}-related projects have emerged over the last years. 
For example, DeepQA of IBM Watson~\cite{watson}, which was able to win the Jeopardy! challenge against human experts. 
Further, {KAIST's Exobrain\footnote{\url{http://exobrain.kr/}}} project aims to learn from large amounts of data while ensuring a natural interaction with end users. 
For further insights, please refer to surveys on existing \ac{QA} approaches~\cite{qasurvey,Kolomiyets:2011,DBLP:journals/semweb/LopezUSM11}.
In Chapter~\ref{cha:hawk}, we present HAWK, our approach towards hybrid \ac{QA}. 
%The field of HAWK refers to hybrid \ac{QA} over \ac{LD}, i.e., \ac{QA} based on hybrid data (RDF and textual data).



